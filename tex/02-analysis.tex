\chapter{Аналитический раздел}

\section{Анализ существующих решений}

Информационные системы для автоматизации продажи вина предоставляют компании-ритейлеры в сфере виноторговли. Существующие программные продукты представляют собой интернет-магазины алкогольных напитков. Сервисы имеются как на российском, так и на зарубежном рынках, и обладают различными функциональностями, отвечающими интересам пользователей. Для анализа были выбраны некоторые из программных решений российского и зарубежного рынков для реализации винодельческой продукции.

\subsection{Российский рынок}

Российские специализированные сети <<ВинЛаб>> \cite{winelab}, <<Красное\&Белое>> \cite{kb} и другие предоставляют веб-сайты для покупки алкогольной продукции. Главная страница сервиса <<ВинЛаб>>, содержащая ссылки на страницы с различными товарами, представлена на рисунке \ref{img:winelab}.

\includeimage
    {winelab}
    {f}
    {h}
    {1.0\textwidth}
    {Главная страница сервиса <<ВинЛаб>>}

На рисунке \ref{img:redAndWhite} показаны различные разделы сайта и каталог интернет-магазина <<Красное\&Белое>>.

\includeimage
    {redAndWhite}
    {f}
    {h}
    {0.7\textwidth}
    {Разделы и каталог веб-сайта <<Красное\&Белое>>}
    
Пользователям российских программных решений доступны следующие действия:
\begin{itemize}
	\item просмотр информации о спиртных напитках;
	\item поиск товаров по различным параметрам;
	\item составление рейтинга продуктов;
	\item регистрация в системе аккаунта покупателя;
	\item получение бонусной карты;
	\item совершение покупок.
\end{itemize}

В каталогах представлена информация не только о винах, но и о других алкогольных напитках. Функциональности сервисов рассчитаны на интересы покупателей.

\subsection{Зарубежный рынок}

Зарубежные программные обеспечения предоставляют компании-ритейлеры <<Primal Wine>> \cite{primal_wine} и <<Wine.com>> \cite{wine_com}. В системах реализованы следующие возможности пользователей:
\begin{itemize}
	\item просмотр сведений о винах;
	\item фильтрация продуктов;
	\item составление рейтинга товаров;
	\item создание аккаунта покупателя;
	\item совершение покупок.
\end{itemize}

Кроме того, <<Primal Wine>> предлагает пользователям стать участником винного клуба. Страница, реализующая возможность вступления в винный клуб, показана на рисунке \ref{img:primalWine}.

\includeimage
    {primalWine}
    {f}
    {h}
    {0.8\textwidth}
    {Страница сервиса <<Primal Wine>>, реализующая возможность вступления в винный клуб}
    
На веб-странице, показанной на рисунке \ref{img:wineCom}, <<Wine.com>> обращается к поставщикам винодельческой продукции, которые могут связаться с компанией-ритейлером для поставки товаров.

\includeimage
    {wineCom}
    {f}
    {h}
    {0.8\textwidth}
    {Страница веб-сайта <<Wine.com>>, содержащая информацию для поставщиков винодельческой продукции}
    
\subsection{Сравнение существующих решений}

Описанные пользовательские возможности (покупка, просмотр и поиск информации, составление рейтинга, регистрация) анализируемых программных решений относятся к функциональностям покупателя. С учетом проведенного обзора можно выделить следующие критерии сравнения существующих сервисов:
\begin{itemize}
	\item К1 --- наличие функциональностей для покупателей;
	\item К2 --- наличие функциональностей для поставщиков;
	\item К3 --- направленность на русскоязычную аудиторию.
\end{itemize}

Результаты сравнения существующих решений по введенным критериям представлены в таблице \ref{tab:comparison}.

\begin{table}[h]
    \caption{Сравнение существующих систем автоматизации продажи вина}
    \begin{center}
        \begin{tabular}{|l|l|l|l|}
            \hline
            \textbf{Решение} & \textbf{К1} & \textbf{К2} & \textbf{К3} \\ \hline
            ВинЛаб & + & - & + \\ \hline
            Красное\&Белое & + & - & + \\ \hline
            Primal Wine & + & - & - \\ \hline
            Wine.com & + & - & - \\ \hline
        \end{tabular}
    \end{center}
    \label{tab:comparison}
\end{table}

\subsection*{Вывод}

В результате сравнения было получено, что как на российском, так и на зарубежном рынке существуют сервисы, автоматизирующие продажу вин. Существующие системы рассчитаны только на покупателей и не предоставляют функциональности для поставщиков. Возможной причиной отсутствия функциональных возможностей для поставщиков является угроза несанкционированной модификации информации.

\section{Проблема нарушения целостности данных}

Поставщик обладает большим уровнем прав доступа, чем покупатель: помимо получения информации о товаре, поставщик должен иметь возможность изменять данные о своей продукции, добавлять и удалять товар. При повышении прав доступа возникает опасность нарушения целостности информации --- неразрешенное изменение, удаление данных или предоставление недействительной информации. Эту проблему можно решить следующим образом: администратор системы должен выдавать права поставщикам на действия, связанные с модификацией данных.

\section{Формализация задачи}

Процесс продажи вина состоит из трех основных этапов.

\begin{enumerate}
	\item Поставщик продает вино определенного сорта, цвета, объема и других параметров ритейлеру по закупочной цене $P_{s}$;
	\item Ритейлер выставляет на продажу полученный товар по цене $S$. Цена $S$ называется ценой реализации товара и формируется путем сложения закупочной цены $P_{s}$ и наценки $N$ \cite{pricing}:
\begin{equation}
    S = P_{s} + N;
\end{equation}
Наценка состоит из издержек (оплата услуг по хранению, операций по приведению товара в удобный для продажи вид) и чистого дохода, в который включаются прибыль и налоги \cite{pricing}. 
	\item Покупатель приобретает вино по цене реализации товара $S$, установленной ритейлером. Поставщик получает часть полученной суммы, равную $P_{s}$. Оставшаяся часть уходит на оплату издержек продажи (налоги, оплата труда, другие материальные расходы) $C$. Таким образом, прибыль ритейлера $P_{r}$ формируется следующим образом:
	
\begin{equation}
    P_{r} = S - P_{s} - C,
\end{equation}
\begin{equation}
    P_{r} = P_{s} + N - P_{s} - C = N - C.
\end{equation}
\end{enumerate}

Входными данным для процесса виноторговли является структура продукта, выходными --- структура продажи.

\subsection{Продукт виноторговли}

Параметры винного продукта могут расширяться в каждом конкретном случае, но основными параметрами являются:

\begin{enumerate}
	\item Сорт;
	\item Цвет;
	\item Объем;
	\item Содержание алкоголя;
	\item Сахар;
	\item Выдержка --- процесс вызревания вина.
\end{enumerate}

Пример структуры вина с введенными параметрами показан в таблице \ref{tab:wine_structure}.

\begin{table}[h]
    \caption{Структура продукта виноторговли}
    \begin{center}
        \begin{tabular}{|l|l|}
            \hline
            \textbf{Параметр} & \textbf{Вино} \\ \hline
            Сорт & Ламбруско \\ \hline
            Цвет & Белое \\ \hline
            Объем (л) & 0.75 \\ \hline
            Содержание алкоголя (\%) & 8 \\ \hline
            Сахар & Полусладкое \\ \hline
            Выдержка (год) & 2 \\ \hline
        \end{tabular}
    \end{center}
    \label{tab:wine_structure}
\end{table}

\subsection{Продажа}

Для учета всех составляющих процесса продажи структура должна содержать следующие параметры:

\begin{enumerate}
	\item Идентификатор вина поставщика;
	\item Идентификатор покупки;
	\item Закупочная цена;
	\item Цена реализации;
	\item Наценка;
	\item Сумма издержек;
	\item Прибыль;
	\item Количество продаваемого товара;
	\item Дата продажи.
\end{enumerate}

Пример структуры продажи с выделенными параметрами показан в таблице \ref{tab:sale_structure}. Параметры №3-№7 указаны в рублях.

\begin{table}[h]
    \caption{Структура продажи}
    \begin{center}
        \begin{tabular}{|l|l|}
            \hline
            \textbf{Параметр} & \textbf{Продажа} \\ \hline
            Идентификатор вина поставщика & 3 \\ \hline
            Идентификатор покупки & 5 \\ \hline
            Закупочная цена & 500 \\ \hline
            Цена реализации & 650 \\ \hline
            Наценка & 150 \\ \hline
            Сумма издержек & 50 \\ \hline
            Прибыль & 100 \\ \hline
            Количество продаваемого товара & 1 \\ \hline
            Дата продажи & 09.09.2022 \\ \hline
        \end{tabular}
    \end{center}
    \label{tab:sale_structure}
\end{table}

\section{Формализация ролей}

Участниками виноторговли, которые будут использовать информационную систему, являются поставщик вин и покупатель. Для выдачи поставщику прав на создание, удаление и изменение вин необходим администратор. Кроме того, до входа в аккаунт или регистрации в системе пользователь является неавторизованным и обладает следующими возможностями:

\begin{itemize}
	\item регистрация аккаунта поставщика или покупателя;
	\item вход в аккаунт;
	\item получение данных о винах.
\end{itemize}

Диаграмма вариантов использования для неавторизованного пользователя представлена на рисунке \ref{img:guest}.

\includeimage
    {guest}
    {f}
    {h}
    {0.7\textwidth}
    {Use-case диаграмма для неавторизованного пользователя}

Для поставщика определены следующие действия:

\begin{itemize}
	\item выход из аккаунта поставщика;
	\item получение данных:
		\begin{itemize}
			\item о винах;
			\item о продажах;
		\end{itemize}		 
	\item добавление нового товара при наличии прав;
	\item удаление товара при наличии прав;
	\item редактирование товара при наличии прав.
\end{itemize}

На рисунке \ref{img:supplier} показана диаграмма вариантов использования для поставщика.

\includeimage
    {supplier}
    {f}
    {h}
    {0.6\textwidth}
    {Use-case диаграмма для поставщика}

В возможности покупателя входит:

\begin{itemize}
	\item выход из аккаунта покупателя;
	\item удаление аккаунта покупателя;
	\item получение данных:
		\begin{itemize}
			\item о винах;
			\item о поставщиках;
			\item о рейтинге вин;
			\item о покупках;
		\end{itemize}
	\item покупка вина;
	\item отмена покупки вина;
	\item получение бонусной карты.	     
\end{itemize}

Диаграмма вариантов использования для покупателя представлена на рисунке \ref{img:customer}.

\includeimage
    {customer}
    {f}
    {h}
    {0.6\textwidth}
    {Use-case диаграмма для покупателя}

Администратор обладает правами на следующие действия:

\begin{itemize}
	\item вход в аккаунт;
	\item выход из аккаунта;
	\item удаление аккаунта поставщика или покупателя;
	\item получение данных:
		\begin{itemize}
			\item о винах;
			\item о продажах;
		\end{itemize}
	\item выдача прав поставщику;
	\item отмена выдачи прав поставщику.
\end{itemize}

На рисунке \ref{img:administrator} показана диаграмма вариантов использования для администратора.

\includeimage
    {administrator}
    {f}
    {h}
    {0.7\textwidth}
    {Use-case диаграмма для администратора}

\section{Формализация данных}

С учетом выделенных структур данных и типов пользователей разрабатываемая база данных должна содержать информацию о следующих данных:

\begin{itemize}
	\item вина;
	\item поставщики;
	\item покупатели;
	\item продажи;
	\item бонусные карты покупателей;
	\item покупки покупателей;
	\item пользователи.
\end{itemize}

ER-модель в нотации Чена, показывающая сущности, их атрибуты и связи между сущностями в разрабатываемой базе данных показа на рисунке \ref{img:chen}.

\includeimage
    {chen}
    {f}
    {h}
    {1.0\textwidth}
    {ER-модель в нотации Чена}

\section{Анализ баз данных}

База данных --- это доступная интегрированная структура, которая состоит из двух частей: данные (необработанные факты) и метаданные (данные о данных), с помощью которых данные интегрируются и управляются \cite{db}.

Система управления базами данных (СУБД) --- это программный комплекс, обеспечивающий централизованное хранение данных и предоставляющий приложениям услуги по обработке данных \cite{dms}.

Существует три группы моделей баз данных, отличающиеся структурой организации данных --- дореляционные, реляционные и постреляционные базы данных.

\subsection{Дореляционные базы данных}

Ранние модели баз данных называются дореляционными \cite{before}. К ним относятся:

\begin{itemize}
	\item иерархическая --- модель представления данных в виде упорядоченного графа (дерева);
	\item сетевая --- модель данных, которая позволяет отображать взаимосвязи элементов данных в виде произвольного графа;
	\item модель, основанная на инвертированных списках.
\end{itemize}

Основным преимуществом дореляционных баз данных является эффективное использование памяти компьютера. Недостатками дореляционных моделей данных являются громоздкость для обработки информации и сложность понимания.

\subsection{Реляционные базы данных}

Реляционные базы данных основаны на понятии отношение \cite{before}. Формой представления отношения являются таблица. Строка таблицы представляет собой запись с уникальным идентификатором --- ключом, и содержит сведения о конкретном предмете. Столбцы таблицы содержат атрибуты данных --- параметры предмета.

Для связывания информации из разных таблиц используются внешние ключи --- уникальные идентификаторы атомарного фрагмента данных в этой таблице. Другие таблицы могут ссылаться на этот внешний ключ, чтобы создать связь между частями данных и частью, на которую указывает внешний ключ.

Реляционные базы данных обеспечивают набор свойств ACID:

\begin{itemize}
	\item атомарность --- транзакция должна выполняться полностью или не выполняться совсем;
	\item непротиворечивость --- по завершении транзакции данные должны соответствовать схеме базы данных;
	\item изолированность --- параллельные транзакции должны выполняться отдельно друг от друга;
	\item надежность --- способность восстанавливаться до последнего сохраненного состояния после непредвиденного сбоя в системе или перебоя в подаче питания.
\end{itemize}

Преимущества реляционных баз данных:

\begin{itemize}
	\item интуитивно понятный, наглядный способ представления данных;
	\item простота установки взаимосвязи между элементами данных;
	\item эффективная поддержка целостности и надежности данных.
\end{itemize}

Недостатки реляционных баз данных:

\begin{itemize}
	\item невозможность представить некоторую предметную область в виде таблицы;
	\item низкая скорость доступа к данным;
	\item необходимость размещения данных внутри таблицы и их описания до начала обработки и установления ограничения на тип данных.
\end{itemize}

\subsection{Постреляционные базы данных}

Для постреляционных моделей данных снимается ограничение неделимости данных: появляется возможность многозначных полей \cite{after}. Набор подзначений становится самостоятельной таблицей, встроенной в главную. Информация может быть представлена с помощью следующих структур организации данных:

\begin{itemize}
	\item хэш-таблица пар <<ключ-значение>>;
	\item документы, упорядоченные по группам, называемым коллекциями;
	\item граф --- модель на основе узлов и ребер, представляющих взаимосвязанные данные.
\end{itemize}

Преимущества постреляционных баз данных:

\begin{itemize}
	\item эффективная масштабируемость;
	\item отсутствие ограничений на типы данных.
\end{itemize}

Недостатки постреляционных баз данных:

\begin{itemize}
	\item несовместимость с запросами SQL;
	\item ограничение требований свойств ACID.
\end{itemize}

\subsection{Выбор структуры организации данных}

В соответствии с формализацией задачи можно выделить следующие требования к разрабатываемой базе данных:

\begin{itemize}
	\item обеспечение надежности;
	\item обеспечение целостности данных;
	\item необходимость сложных запросов.
\end{itemize}

Исходя из преимуществ и недостатков типов баз данных и требований к разрабатываемой базе данных можно сделать вывод о том, что для решения задачи необходимо разработать и использовать реляционную базу данных.

\section*{Вывод}

В данном разделе были описаны структуры вина и продажи, выделены возможности пользователей и определены категории данных. В результате анализа существующих решений не было найдено полноценных аналогов. По результатам сравнения моделей баз данных для реализации была выбрана реляционная модель данных.
