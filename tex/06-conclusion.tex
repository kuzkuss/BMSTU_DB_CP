\chapter*{ЗАКЛЮЧЕНИЕ}
\addcontentsline{toc}{chapter}{ЗАКЛЮЧЕНИЕ}

В ходе выполнения курсовой работы были разработаны база данных вин и приложение, которое предоставляет функциональные возможности для всех участников процесса виноторговли: поставщиков и покупателей.

Проведенный обзор существующих решений показал, что у представленных программных продуктов в России и за рубежом отсутствуют функциональности для поставщиков. В результате формального описания процесса продажи вина были определены типы пользователей программного обеспечения и их возможные действия. На основе требований к базе данных и анализе структур организации данных была выбрана реляционная модель данных.

При проектировании базы данных были описаны поля таблиц. Для обеспечения ссылочной целостности таблиц базы данных были разработаны два DML-триггера. Для поддержки безопасной обработки запросов к таблицам были определены права для различных ролей пользователей. При разработке структуры информационной системы были выделены уровни бизнес-логики, доступа к данным и представления.

Исходя из сравнения СУБД для управления базой данных была выбрана PostgreSQL --- СУБД с открытым исходным кодом, поддерживающая соответствие свойствам ACID и создание триггеров. Реализация программного обеспечения проводилась на языке программирования C\# с использованием фреймворков Entity Framework Core и Blazor и среды разработки Visual Studio.

Также был проведен эксперимент, в результате которого было выяснено, что использование индексирования повышает скорость выполнения поиска в таблице базы данных по внешнему ключу. Наличие индекса при поиске по первичному ключу не увеличивает производительность обработки запроса, так как для первичных ключей индексы создаются автоматически.

Таким образом, поставленные задачи были выполнены, цель курсовой работы была достигнута.

Возможное дальнейшее развитие системы связано с поддержкой целостности данных. Решением этой задачи может быть добавление следующей функциональности: администратор должен одобрять или отклонять запросы поставщиков на модификацию данных о винодельческой продукции. Еще одним направлением дальнейшего развития является добавление большего числа функциональностей для учета продаж.